\documentstyle[12pt]{article}
%\documentstyle[12pt,my]{article}
\addtolength{\oddsidemargin}{0in}
\addtolength{\textwidth}{+0.5in}
\addtolength{\topmargin}{-0.5in}       % slightly longer page
\addtolength{\textheight}{0.5in}
\renewcommand{\topfraction}{0.95}
\renewcommand{\textfraction}{0.05}  %should make [h] work as desired
\parindent 0pt

%%%\renewcommand{baselinestretch}{1.2}

\newcommand{\mysk}{\vspace{0.5cm}}

\title{\vspace{2cm}\goodbreak 
\bf {\sl Aflex} \rm -- An Ada Lexical Analyzer Generator
\\ \vspace{1cm} Version 1.1 \vspace{1cm}
}
\author{\large \rm John Self \\
\ \\
Arcadia Environment Research Project \\
Department of Information and Computer Science\\
University of California, Irvine \\
\\UCI-90-18\\
\medskip\\
Adapted for the GNU Ada Compiler GNAT\\
by the ``Ada for Linux Team (ALT)'', Feb 1999\\
\\
\thanks{This work was supported in
part by the National Science Foundation under grants CCR--8704311
and CCR--8451421 with cooperation from the Defense Advanced Research
Projects Agency, and by the National Science Foundation under Award
No. CCR-8521398.}
}

\date{May 1990}

\begin{document}

\maketitle

\begin{titlepage}
\tableofcontents
\end{titlepage}

\section{Introduction}
\label{intro}
{\sl Aflex} is a lexical analyzer generating tool written in Ada
designed for lexical 
processing of character input streams.
It is a successor to the {\sl Alex}\cite{alex} tool from UCI.  {\sl Aflex}
is upwardly compatible with {\sl alex 1.0}, but is significantly
faster at generating scanners, and produces smaller scanners for
equivalent specifications.  Internally {\sl aflex} is patterned after the
{\sl flex} tool from the GNU project.
{\sl Aflex} accepts high level rules  written in regular expressions
for character string matching, and generates Ada source code comprising a
lexical analyzer along with two auxiliary Ada packages.  The main file
includes a routine that partitions the input text stream into strings 
matching the expressions.  Associated with each rule is an 
action block composed of program fragments.  Whenever a rule is recognized 
in the input stream, the corresponding program fragment is executed.
This feature, combined with the powerful string pattern matching capability,
allows the user to implement a lexical analyzer for any type of application
efficiently and quickly.
For instance, {\sl aflex} can be used alone for simple lexical analysis and 
statistics, or with {\sl ayacc} \cite{ayacc} to generate a parser front-end.  
{\sl Ayacc} is an Ada parser generator that accepts context-free grammars.

\mysk
{\sl Aflex} is a successor to the Arcadia tool {\sl Alex}\cite{alex} which
was inspired by the popular Unix operating system tool, {\it lex} 
\cite{lex},.  Consequently, most of {\it lex}'s features and conventions are 
retained in {\sl aflex}; however, a few important differences are discussed 
in section \ref{lexdiff}.  There are also a few minor differences
between {\sl aflex} and {\sl alex} which will be discussed in
section \ref{alexdiff}.

\mysk
This paper is intended to serve as both the reference manual and the
user manual for {\sl aflex}.  Some knowledge of {\it lex}, while not
required, would be very useful in understanding the use of{\sl aflex.}
A good introduction to {\it lex}, as well as lexical and syntactic analysis, 
can be found in \cite{dragon}, frequently referred to as ``the Dragon Book.''
Topics to be covered in this paper include the usage of
{\sl aflex}, the operators' description, the source file format,
the generated output, the necessary interfaces with {\sl ayacc},
and ambiguity among rules.
The appendices provide a simple example, {\sl aflex} dependencies,
the differences between {\sl aflex},{\sl alex}, and {\it lex}, known bugs and
limitations, and references.

\newpage
\section{Command Line Options}
Command line options are given in a different format than in the
old UCI alex.  Aflex options are as follows
\begin{description}
\item[-t]
Write the scanner output to the standard output rather than to a file.
The default names of the scanner files for base.l are base.ads and base.adb.
Note that this option is not as useful with aflex because in addition 
to the scanner file there are files for the externally visible dfa functions
(base-dfa.ad[sb]) and the external IO functions (base-io.ad[sb])
\item[-b]
Generate backtracking information to
{\it} aflex.backtrack.
This is a list of scanner states which require backtracking
and the input characters on which they do so.  By adding rules one
can remove backtracking states.  If all backtracking states
are eliminated and
{\bf -f}
is used, the generated scanner will run faster (see the
{\bf -p}
flag).  Only users who wish to squeeze every last cycle out of their
scanners need worry about this option.
\item[-d]
makes the generated scanner run in
{\it debug}
mode.  Whenever a pattern is recognized the scanner will
write to
{\it stderr}
a line of the form:
\begin{verbatim}

    --accepting rule #n

\end{verbatim}
Rules are numbered sequentially with the first one being 1.  Rule \#0
is executed when the scanner backtracks; Rule \#(n+1) (where
{\it n}
is the number of rules) indicates the default action; Rule \#(n+2) indicates
that the input buffer is empty and needs to be refilled and then the scan
restarted.  Rules beyond (n+2) are end-of-file actions.
\item[-f]
has the same effect as lex's -f flag (do not compress the scanner
tables); the mnemonic changes from
{\it fast compilation}
to (take your pick)
{\it full table}
or
{\it fast scanner.}
The actual compilation takes
{\it longer,}
since aflex is I/O bound writing out the big table.
The compilation of the Ada file containing the scanner is also likely
to take a long time because of the large arrays generated.
\item[-i]
instructs aflex to generate a
{\it case-insensitive}
scanner.  The case of letters given in the aflex input patterns will
be ignored, and the rules will be matched regardless of case.  The
matched text given in
{\it yytext}
will have the preserved case (i.e., it will not be folded).
\item[-p]
generates a performance report to stderr.  The report
consists of comments regarding features of the aflex input file
which will cause a loss of performance in the resulting scanner.
Note that the use of
the
{\bf \verb|^|}
operator
and the
{\bf -I}
flag entail minor performance penalties.
\item[-s]
causes the
{\it default rule}
(that unmatched scanner input is echoed to
{\it stdout)}
to be suppressed.  If the scanner encounters input that does not
match any of its rules, it aborts with an error.  This option is
useful for finding holes in a scanner's rule set.
\item[-v]
has the same meaning as for lex (print to
{\it stderr}
a summary of statistics of the generated scanner).  Many more statistics
are printed, though, and the summary spans several lines.  Most
of the statistics are meaningless to the casual aflex user, but the
first line identifies the version of aflex, which is useful for figuring
out where you stand with respect to patches and new releases.
\item[-E]
instructs aflex to generate additional information about each token,
including line and column numbers.  This is needed for the advanced
automatic error option correction in ayacc.
\item[-I]
instructs aflex to generate an
{\it interactive}
scanner.  Normally, scanners generated by aflex always look ahead one
character before deciding that a rule has been matched.  At the cost of
some scanning overhead, aflex will generate a scanner which only looks ahead
when needed.  Such scanners are called
{\it interactive}
because if you want to write a scanner for an interactive system such as a
command shell, you will probably want the user's input to be terminated
with a newline, and without
{\bf -I}
the user will have to type a character in addition to the newline in order
to have the newline recognized.  This leads to dreadful interactive
performance.

If all this seems to confusing, here's the general rule: if a human will
be typing in input to your scanner, use
{\bf -I,}
otherwise don't; if you don't care about how fast your scanners run and
don't want to make any assumptions about the input to your scanner,
always use
{\bf -I.}

Note,
{\bf -I}
cannot be used in conjunction with
{\it full}
i.e., the
{\bf -f}
flag.
\item[-L]
instructs aflex to not generate
{\bf \#line}
directives (see below).
\item[-T]
makes aflex run in
{\it trace}
mode.  It will generate a lot of messages to stdout concerning
the form of the input and the resultant non-deterministic and deterministic
finite automatons.  This option is mostly for use in maintaining aflex.
\item[-Sskeleton\_file]
overrides the default internal skeleton from which aflex constructs
its scanners.  You'll probably never need this option unless you are doing
aflex maintenance or development.
\end{description}
\section{{\sl Aflex} Output}
{\sl Aflex} generates a file containing a lexical analyzer function along
with two auxiliary packages, all of which are written in Ada.  
The context in which the lexical analyzer function is defined is flexible 
and may be specified by the user.  For instance,  the file may only 
contain the lexical analyzer function as a single compilation unit which 
may be called by {\sl ayacc},  
or it may be placed within a package body or embedded within a driver 
routine.  This scanner function, when invoked, partitions the character stream 
into tokens as specified by the regular expressions defined in the rules 
section of the source file.  The name of the lexical analyzer 
function is {\sl yylex}. Note that it returns values of type {\it token}.  
Type {\it token} must be defined as an enumeration type which contains, 
at a minimum, ({\it End\_of\_Input, Error}). It is up to the user to make 
sure that this type is visible (see Section \ref{alexayacc}). The general 
format of the output file which contains this function is found in
Figure 3.

\mysk
The auxiliary packages include a DFA and an IO package.  The DFA
package contains externally visible functions and variables from the
scanner.  Many of the variables in this package should not be modified
by normal user programs, but they are provided here to allow the user to
modify the internal behavior of aflex to match specific needs.  Only
the functions YYText and YYLength will be needed by most programs. The
{\sl GNAT} port of {\sl aflex} generates the DFA package as child packages
of the base package. For portability and conveniance the previously
used flat package names  {\sl base}\_DFA and {\sl base}\_IO are generated
as renames of these child packages.

\mysk
The IO package contains
routines which allow {\sl yylex} to scan the input source file.
These include the unput, input, output, and yywrap functions
from {\it lex},\\
plus Open\_Input, Create\_Output, Close\_Input and Close\_Output
provided for compatibility with {\sl alex.}
\mysk
It is also possible to write your own IO and DFA packages.  Redefining
input is possible by changing the YY\_INPUT procedure.  As an example
you might wish to take input from an array instead of from a file.  By
changing the calls to the TEXT\_IO routines to access elements of the
array you can change the input strategy.  If you change the IO or DFA
packages you should make a copy of the generated files under a
different name and change that, because {\sl aflex} will overwrite
them whenever you rerun {\sl aflex}.

\newpage
\small
\begin{tabbing}
1234\=1234\=1234\=1234\=1234\=1234 \kill

    \>\>    \>{\bf with} $<$rootname$>$\.DFA; \\
    \>\>    \>{\bf with} $<$rootname$>$\.IO; \\
    \>\>    \>{\bf with} TEXT\_IO; \\
\\
    \>\>    \>\verb|--| User Specified Context\\
\\
    \>\>    \>    \>{\bf function} yylex {\bf return} Token {\bf is} \\
    \>\>    \>    \>{\bf begin} \\
    \>\>    \>    \>    \>\verb|--| Analysis of expressions \\
    \>\>    \>    \>    \>\verb|--| Execution of user-defined actions \\
    \>\>    \>    \>{\bf end} yylex; \\
\\
    \>\>  \>\verb|--| User Specified Context\\
\end{tabbing}
\centerline{Figure 3: Example of File Containing Lexical Analyzer}

\mysk
Before showing the general layout of the specification file, we will
describe the specification language of {\sl aflex}, namely, regular expressions.


\section{Regular Expressions}
{\sl Aflex} distinguishes two types  of  character  sets  used  to
define regular expressions: text characters and operator characters.
A regular expression specifies how a set of strings from the input
string can be recognized.  It contains text characters  (which  match  the
corresponding characters in  the  strings  being  compared)  and
operator  characters  (which  specify  repetitions, choices, and
other features).  The letters of the alphabet  and  the  digits  are
always text characters.

\mysk
A rule specifies a  sequence of  characters  to  be  matched. It 
{\bf must} begin in column one. 
The set of {\sl aflex} operators consists of 
the following:

\begin{verbatim}
      " \ { } [ ] ^ $ < > ? . * + | ( ) /   
\end{verbatim}

The meaning of each operator is summarized below:

\begin{tabbing}
1234\=1234\=1234\=1234\=1234\=1234 \kill
  \>\verb|x|     \>\>\verb|--| the character ``x" \\
  \>\verb|"x"|   \>\>\verb|--| an ``x", even if x is an operator. \\
  \>\verb|\x|    \>\>\verb|--| an ``x", even if x is an operator. \\
  \>\verb|^x|    \>\>\verb|--| an x at the beginning of a line. \\
  \>\verb|x$|    \>\>\verb|--| an x at the end of line. \\
  \>\verb|x+|    \>\>\verb|--| 1 or more instances of x. \\
  \>\verb|x*|    \>\>\verb|--| 0 or more instances of x. \\
  \>\verb|x?|    \>\>\verb|--| an optional x. \\
  \>\verb|(x)|   \>\>\verb|--| an x. \\
  \>\verb|.|     \>\>\verb|--| any character but newline. \\
  \>\verb"x|y"   \>\>\verb|--| an x or y. \\
  \>\verb|[xy]|  \>\>\verb|--| the character x or the character y. \\
  \>\verb|[x-z]| \>\>\verb|--| the character x, y or z. \\
  \>\verb|[^x]|  \>\>\verb|--| any character but x. \\
  \>\verb|<y>x|  \>\>\verb|--| an x when {\sl aflex} is in start condition y. \\
  \>\verb|{xx}|  \>\>\verb|--| the translation of xx from the definitions section. \\
\end{tabbing}

If any of these operators is used in a regular expression as a character
literal, it must be either preceded by an escape character or surrounded by
double quotes.  For example, to recognize a dollar sign \verb|$|, the correct
expression is either \verb|\$| or \verb|"$"|.
Note a quote cannot be quoted and should therefore be escaped.

\mysk
A regular expression may {\bf not} contain any spaces
unless they are within in a quoted string or character class 
or they are preceded by the \verb|"\"| operator. 

\mysk
When in doubt, use parentheses.  When an {\sl aflex} operator needs to be 
embedded in a string, it is often neater to quote the entire string rather 
than just the operator, e.g. the string \verb|"what?"| is more readable 
than both \verb|What"?"|, and \verb|What\?|.

\small
\begin{verbatim}
Rules              Interpretations
-----              ---------------
a or "a"           The character a
Begin or "Begin"   The string Begin
\"Begin\"          The string "Begin"
^\t or ^"\t"       The tab character \t at the beginning of line.
\n$                The newline character \n at the end of line.
\end{verbatim}
\normalsize

There are a few special characters which can be specified in a regular
expression:
\begin{tabbing}
1234\=1234\=1234\=1234\=1234\=1234 \kill
  \>\verb|\n|     \>\>\verb|--| newline \\
  \>\verb|\b|     \>\>\verb|--| backspace \\
  \>\verb|\t|     \>\>\verb|--| tab \\
  \>\verb|\r|     \>\>\verb|--| carriage return \\
  \>\verb|\f|     \>\>\verb|--| form feed \\
  \>\verb|\ddd|   \>\>\verb|--| octal ASCII code \\
\end{tabbing}
Here is the precedence of the above  operators  that  have  precedence. 
\begin{tabbing}
1234\=1234\=1234\=1234\=1234\=1234 \kill
  \>\verb|"  []  ()|            \>\>\>\>Highest \\
  \>\verb|+  *  ?|		\>\>\>\>\hspace{0.5cm}$\vdots$ \\
  \>\verb|concatenation|	\>\>\>\>\hspace{0.5cm}$\vdots$ \\
  \>\verb"|"                    \>\>\>\>Lowest \\
\end{tabbing}

\begin{description}
  \item[Character Classes:]  Classes of characters can be specified using
      the  operator pair {\bf []}.  Within these square brackets, the operator
      meanings are ignored except for three special characters:  \verb|\|
      and $-$ and \verb|^|.

\small
\begin{verbatim}
Rules       Interpretations
-----       ---------------
[^abc]      Any character except a, b, or c.
[abc]       The single character a, b, or c.
[-+0-9]     The - or + sign or any digit from 0 to 9.
[\t\n\b]    The tab, newline, or backspace character.
\end{verbatim}
\normalsize

  \item[Arbitrary and Optional Characters:]  The dot, ``$.$", operator 
  matches all characters except newline.  The  operator  ?  indicates  an
  optional character of an expression.

\small
\begin{verbatim}
Rules     Interpretations
-----     ---------------
ab?c      Matches either abc or ac.
ab.c      Matches all strings of length 4 having a, b and
          c as the first, second and fourth letter where the
          third character is not a newline.
\end{verbatim}
\normalsize


  \item[Repeated Expressions:]  Repetitions of classes are  indicated  by
      the operators $*$ and $+$.

\small
\begin{verbatim}
Rules                  Interpretations
-----                  ---------------
[a-z]+                 Matches all strings of lower case letters.
[A-Za-z][A-Za-z0-9]*   Indicates all alphanumeric strings with a
                       leading alphabetic character.
\end{verbatim}
\normalsize


  \item[Alternation and Grouping:]  The operator \verb"|" indicates  alternation
      and parentheses are used for grouping complex expressions.

\small
\begin{verbatim}
Rules           Interpretations
-----           ---------------
ab|cd           Matches either ab or cd.
(ab|cd+)?(ef)*  Matches such strings as abefef, efefef, cdef,
                or cddd; but not abc, abcd, or abcdef.
\end{verbatim}
\normalsize


  \item[Context Sensitivity:]  {\sl aflex} will recognize a small amount of
      surrounding context.  Two simple operators for this are \verb|^| and
      \$.  If the first character of an expression is \verb|^|, the expression
      will  only  be  matched at the beginning of a line.  If the very
      last character is \$, the expression will only be matched at  the
      end  of  a line.  

\small
\begin{verbatim}
Rules          Interpretations
-----          ---------------
^ab            Matches ab at the beginning of line.
ab$            Matches ab at the end of line.
\end{verbatim}
\normalsize


  \item[Definitions:]  The operators  \{ \}  enclosing a name
      specify a macro definition expansion.

\small
\begin{verbatim}
Rules          Interpretations
-----          ---------------
{INTEGER}      If INTEGER is defined in the macro definition
               section, then it will be expanded here.
\end{verbatim}
\normalsize
\end{description}


\subsection{Predefined Variables \& Routines}
\label {routines}
Once a token is matched, the textual string representation of the token
may be obtained by a call to the function {\sl yytext} which is located
in the {\sl dfa package}.  This function returns type string.

\mysk
The IO package contains
routines which allow {\sl yylex} to scan the input source file.
These include the input, output, unput and yywrap functions
from lex,\\ plus Open\_Input, Create\_Output, Close\_Input and Close\_Output
provided for compatibility with {\sl alex.}  Note that in {\sl alex
1.0} it was mandatory to call the {\it Open\_Input} and {\it
Create\_Output} routines before calling {\it YYLex.}  This is not
required in {\sl Aflex.}  The default input and output are attached to
the files that Ada considers to be the {\sc standard\_input} and
{\sc standard\_output. }

The following routines must be used in lieu of the normal {\sc
text\_io} routines because of internal buffering and read-ahead done by
{\sl aflex.} 

\begin{description}
\item[input] function input return character -- inputs a character from the
current {\sl aflex} input stream.
\item[unput] procedure unput(c : character) -- returns a character
already read by input to the input stream.  Note that attempting to
push back more than one character at a time can cause {\sl aflex} to\\
raise the exception {\sc pushback\_overflow.}
\item[output] procedure output(c : character) -- outputs a character to the
current {\sl aflex} output stream.
\item[yywrap] function yywrap return boolean -- This function is
called when {\sl aflex} reaches the end of file.  If {\it yywrap}
returns true, {\sl aflex} continues with normal wrapup at end of
input.  If you wish to arrange for more input to arrive from a new
source then you provide a yywrap which returns false.  The default
yywrap return true.
\item[Open\_Input] Open\_Input(fname : in String) -- Uses the file named
fname as the source for input to {\it YYLex.}  If this function is not
called then the default input is the Ada {\sc standard\_input.}
\item[Open\_Input] Create\_Output(fname : in String) -- Uses the file named
fname as output for {\it YYLex.}  If this function is not
called then the default output is the Ada {\sc standard\_output}.
\item[Close\_Input and Close\_Output]  These functions have null
bodies in {\sl aflex} and are provided only for compatibility with
{\sl alex.}
\end{description}

\mysk
There are a few predefined subroutines that may be used once a token
is matched.  In many lexical processing applications, the printing of
the string returned by {\sl yytext}, i.e. {\tt put(yytext)}, is desired 
and this action is so common that it may be written as {\tt ECHO}.

\newpage
\section{{\sl Aflex} Source Specification}
\label {specformat}

The general format of the source file is

\small
\begin{verbatim}
    definitions section
    %% 
    rules section
    %% 
    user defined section copied before package statement of SPEC file
    ##
    user defined section copied after package statement of SPEC file
    ##
    user defined section copied before package statement of BODY file
    ##
    user defined section copied after  package statement of BODY file
    but before YYLex
    ##  --  here goes YYLex
    section copied after YYLex and before end of BODY package 
\end{verbatim}
\normalsize

where \verb|%%| is used as a delimiter between sections  and \verb|##| 
indicates where the user supplied code and where function {\sl yylex} 
will be placed.  Both \verb|%%| 
and \verb|##| {\it must} occur in column one.

\mysk
The definitions section is used to define macros which appear in the rules 
section and also to define start conditions.  The rules section defines the 
regular expressions with their corresponding actions.  These regular 
expressions, in turn, define the tokens to be identified by the scanner.
The user defined sections allows the user to define the context in which the 
{\sl yylex} function will be located.  The user can include routines which
may be executed when a certain token or condition is recognized.  


\subsection{Definitions Section}

The definitions section may contain both macro definitions and
start condition definitions.  Macro and start condition definitions
must begin in column one and may be interspersed.

\subsubsection{Macros}
Macro definitions take the form:

\small
\begin{verbatim}
    name   expression
\end{verbatim}
\normalsize

where {\tt name} must begin with a letter and contain only letters,
digits and underscores, and {\tt expression} is
any string of characters that {\tt name} will be textually substituted to 
if found in the rule section.  At least one space must separate {\tt name} 
from {\tt expression} in the definition.  No syntax checking is done in 
the expression, instead the whole rule is parsed after expansion.
The macro facility is very useful in writing regular expressions which
have common substrings, and in defining often-used ranges like {\it digit} 
and {\it letter}.
Perhaps its best advantage is to give a mnemonic name to a rather strange
regular expression -- making it easier for the programmer to debug the
expressions.  These macros, once defined, can be used in the
regular expression by surrounding them with \{ and \}, e.g., \verb|{DIGIT}|.
For example, the rule

\small
\begin{verbatim}
[a-zA-Z]([0-9a-zA-Z])*   {put_line ("Found an identifier");}
[0-9]+                   {put_line ("Found a number");}
\end{verbatim}
\normalsize

defines identifiers and integer numbers.  With macros, the source file is

\small
\begin{verbatim}
LETTER [a-zA-Z]
DIGIT  [0-9]
%%
{LETTER}({DIGIT}|{LETTER})*   {put_line ("Found an identifier");}
{DIGIT}+                      {put_line ("Found a number");}
\end{verbatim}
\normalsize

\mysk
It is customary, although not necessary, to use all capital letters
for macro names. This allows macros to be easily identified in complex rules.
Macro names are case sensitive, e.g., \verb|{DIGIT}| and \verb|{Digit}| are
two different macro names.

\subsubsection{Start Conditions}
Left context is handled in {\sl aflex} by start conditions that are defined 
in the macro definition section.  Start conditions are declared as follows,

\begin{verbatim}
      %Start cond1 cond2 ...
\end{verbatim}

where cond1 and cond2 indicate start conditions.
Note that \%Start may be abbreviated as \%S or \%s.

\mysk
A condition is set only when  the  {\sl aflex}  command  {\tt ENTER}  in  the
action  part  is executed, e.g. {\tt ENTER cond1};. Thus the expression 
which  has  the  form \verb|<condition>rule| will only be matched 
when {\tt condition} is set.  Note that {\sl aflex} uses {\tt ENTER}
instead of {\tt BEGIN} which is used in {\it lex.}  This is done
because {\tt BEGIN} is a keyword in Ada.  The {\tt ENTER} command must
have parentheses surrounding its argument.
\begin{verbatim}
      ENTER(cond1);
\end{verbatim}

{\sl Aflex} also provides {\it exclusive start conditions.}  These are
similar to normal start conditions except they have the property that
when they are active no other rules are active.  Exclusive start
conditions are declared and used like normal start conditions except
that the declaration is done with \%x instead of \%s.

\subsection{Rules Section}

Contained in the rule section are regular expressions which define the
format of each token to be recognized by the scanner.
Each rule has the following format:

\begin{verbatim}
pattern  {action}
\end{verbatim}

where {\tt pattern} is a regular expression and {\tt action} is an Ada 
code fragment enclosed between \{ and \}.  A {\tt pattern} must 
always begin in column one.

\mysk
While a pattern defines the format of the token, the action portion
defines
the operation to be performed by the scanner each time the corresponding
token is recognized.  Therefore, the user must provide a syntactically
correct Ada code fragment.  {\sl aflex} does not check for the validity of the
program portion, but rather copies it to the output package and leaves it to
the Ada compiler to detect syntax and semantics errors.  There can be more 
than one Ada statement in the code fragment.  For example, the rule

\small
\begin{verbatim}
%%
begin|BEGIN     {copy (yytext, buffer);
                 Install (yytext,symbol_table);
                 return RESERVED;}
\end{verbatim}
\normalsize

recognizes the reserved word ``begin" or ``BEGIN", copies the
token string into the buffer, inserts it in the symbol table and returns 
the value, RESERVED.  

Note that the user must provide the procedures
{\tt copy} and {\tt install} along with all necessary types and variables
in the user defined sections.

\subsection{User Defined Sections}
The user defined sections allows the user to specify the context surrounding
the {\sl yylex} function.  \verb|##| is used to separate the various parts
of user defined code and where the {\sl yylex} function should be placed. 
It must be present in this section and must occur in the first column.  
Any text following \verb|##| on the same line is ignored. This method of
using multiple user defined sections that go to specific places in the
generated .ads and .adb files is specific for the {\sl GNAT} port of 
{\sl aflex}.

\section{Ambiguous Source Rules}
When a set of regular expressions is ambiguous, {\sl aflex} uses the
following rules to choose among the regular expressions that match
the input.
\begin{enumerate}
    \item The longest string is matched.
    \item If the strings are of the same length, the rule given
	  {\bf first} is matched.
\end{enumerate}

For example, if input \verb|"aabb"| matches both \verb|"a*"| and
\verb|"aab*"| the action associated with \verb|"aab*"| is executed
because it matches four as opposed to two characters.

\section{{\sl Aflex} and {\sl Ayacc}}
\label{alexayacc}
As briefly mentioned in Section \ref{intro}, {\sl aflex} can be integrated with
{\sl ayacc} to produce a parser.

\mysk
Since the parser generated by {\sl ayacc} expects a value of type {\it token},
each {\sl aflex} rule should end with 

\begin{verbatim}
     return (token_val);
\end{verbatim}

to return the appropriate token value.  {\sl Ayacc} creates a package
defining this token type from its specification file, which in turn 
should be {\it with}'ed at the beginning of the user defined section.
Thus, this token package must be compiled before the lexical analyzer.
The user is encouraged to read the Ayacc User Manual \cite{ayacc} for 
more information on the interaction between {\sl aflex} and {\sl ayacc}.


\newpage
\section{Appendix A: A Detailed Example}

This section shows a complete {\sl aflex} specification file for translating all
characters to uppercase.  The following file,
{\it example.l}, defines rules for recognizing lowercase and uppercase words.
If a word is in lowercase, the scanner converts it to uppercase.
In addition, the frequencies of lower and uppercase words
are retained in the two variables defined in the global section.
All other characters (spaces, tabs, punctuation) remain the same.

\small
\begin{verbatim}
LOWER    [a-z]
UPPER    [A-Z]

%%

{LOWER}+       { Lower_Case := Lower_Case + 1; 
                 TEXT_IO.PUT(To_Upper_Case(Example_DFA.YYText)); }

		  -- convert all alphabetic words in lower case
		  -- to upper case

{UPPER}+       { Upper_Case := Upper_Case + 1;
                 TEXT_IO.PUT(Example_DFA.YYText); }

		  -- write uppercase word as is

\n             { TEXT_IO.NEW_LINE;}

.              { TEXT_IO.PUT(Example_DFA.YYText); }
                 -- write anything else as is

%% -- The next section will go to example.ads before the package statement
with Ada.Command_Line;
## -- The next section will go to example.ads after  the package statement
procedure Sample;
## -- The next section will go to example.adb before the package statement

## -- The next section will go to example.adb after  the package statement
procedure Sample is

  type Token is (End_of_Input, Error);

  Tok        : Token;
  Lower_Case : NATURAL := 0;   -- frequency of lower case words
  Upper_Case : NATURAL := 0;   -- frequency of upper case words

  function To_Upper_Case (Word : STRING) return STRING is
  Temp : STRING(1..Word'LENGTH);
  begin
    for i in 1.. Word'LENGTH loop
      Temp(i) := CHARACTER'VAL(CHARACTER'POS(Word(i)) - 32); 
    end loop;
    return Temp;
  end To_Upper_Case;  
##  -- function YYLex will go here, the follwing lines after YYLex
begin  -- Sample

  Example_IO.Open_Input     (Ada.Command_Line.Argument (1));

  Read_Input :
  loop
    Tok := YYLex;
    exit Read_Input
      when Tok = End_of_Input;
  end loop Read_Input;

  TEXT_IO.NEW_LINE;
  TEXT_IO.PUT_LINE("Number of lowercase words is => " & 
		   INTEGER'IMAGE(Lower_Case));
  TEXT_IO.PUT_LINE("Number of uppercase words is => " & 
		   INTEGER'IMAGE(Upper_Case));
end Sample;
\end{verbatim}
\normalsize

This source file is run through {\sl aflex} using the command

\small
\begin{verbatim}
% aflex example.l
\end{verbatim}
\normalsize
   
{\sl aflex} produces output files called {\it example.ads} and {\it example.adb}
along with two packages, {\it example\-dfa.ads}, {\it example\-dfa.adb}, 
{\it example\-io.ads} and {\it example\-io.adb}.
Assuming that the main procedure, {\sl Sample}, is used to construct
an object file called {\it sample}, the Unix command

\small
\begin{verbatim}
% sample example.l
\end{verbatim}
\normalsize

prints to the screen the exact file {\it example.l} with letters in 
uppercase, i.e. the output to the screen is

\newpage
\small
\begin{verbatim}
LOWER    [A-Z]
UPPER    [A-Z]

%%

{LOWER}+       { LOWER_CASE := LOWER_CASE + 1; 
                 TEXT_IO.PUT(TO_UPPER_CASE(EXAMPLE_DFA.YYTEXT)); }

		  -- CONVERT ALL ALPHABETIC WORDS IN LOWER CASE
		  -- TO UPPER CASE

{UPPER}+       { UPPER_CASE := UPPER_CASE + 1;
                 TEXT_IO.PUT(EXAMPLE_DFA.YYTEXT); }

		  -- WRITE UPPERCASE WORD AS IS

\N             { TEXT_IO.NEW_LINE;}

.              { TEXT_IO.PUT(EXAMPLE_DFA.YYTEXT); }
                 -- WRITE ANYTHING ELSE AS IS

%% -- THE NEXT SECTION WILL GO TO EXAMPLE.ADS BEFORE THE PACKAGE STATEMENT
WITH ADA.COMMAND_LINE;
## -- THE NEXT SECTION WILL GO TO EXAMPLE.ADS AFTER  THE PACKAGE STATEMENT
PROCEDURE SAMPLE;
## -- THE NEXT SECTION WILL GO TO EXAMPLE.ADB BEFORE THE PACKAGE STATEMENT

## -- THE NEXT SECTION WILL GO TO EXAMPLE.ADB AFTER  THE PACKAGE STATEMENT
PROCEDURE SAMPLE IS

  TYPE TOKEN IS (END_OF_INPUT, ERROR);

  TOK        : TOKEN;
  LOWER_CASE : NATURAL := 0;   -- FREQUENCY OF LOWER CASE WORDS
  UPPER_CASE : NATURAL := 0;   -- FREQUENCY OF UPPER CASE WORDS

  FUNCTION TO_UPPER_CASE (WORD : STRING) RETURN STRING IS
  TEMP : STRING(1..WORD'LENGTH);
  BEGIN
    FOR I IN 1.. WORD'LENGTH LOOP
      TEMP(I) := CHARACTER'VAL(CHARACTER'POS(WORD(I)) - 32); 
    END LOOP;
    RETURN TEMP;
  END TO_UPPER_CASE;  
##  -- FUNCTION YYLEX WILL GO HERE, THE FOLLWING LINES AFTER YYLEX
BEGIN  -- SAMPLE

  EXAMPLE_IO.OPEN_INPUT     (ADA.COMMAND_LINE.ARGUMENT (1));

  READ_INPUT :
  LOOP
    TOK := YYLEX;
    EXIT READ_INPUT
      WHEN TOK = END_OF_INPUT;
  END LOOP READ_INPUT;

  TEXT_IO.NEW_LINE;
  TEXT_IO.PUT_LINE("NUMBER OF LOWERCASE WORDS IS => " & 
		   INTEGER'IMAGE(LOWER_CASE));
  TEXT_IO.PUT_LINE("NUMBER OF UPPERCASE WORDS IS => " & 
		   INTEGER'IMAGE(UPPER_CASE));
END SAMPLE;

Number of lowercase words is =>  199
Number of uppercase words is =>  127
\end{verbatim}
\normalsize


\newpage
\section{Appendix B: {\sl Aflex} Dependencies}

This release of {\sl aflex} was successfully compiled by GNAT-3.11p
running under Linux 2.2.x and glibc-2.0 by the Ada for Linux Team (ALT).

\subsection{Command Line Interface}
The following files are host dependent : 
\begin{tabbing}
1234\=1234\=1234\=1234\=1234\=1234 \kill
    \>    \>{\sl command\_lineS.a}\\
    \>    \>{\sl command\_lineB.a}\\
    \>    \>{\sl file\_managerS.a}\\
    \>    \>{\sl file\_managerB.a}\\
\end{tabbing}
The command\_line package function {\sc initialize\_command\_line}
breaks up the command line into a vector containing
the arguments passed to the program.  Note that modifications may need
to be made to this file if the host system doesn't allow differentiation
of upper and lower case on the command line.
\mysk
The external\_file\_manager package is host dependent in that it chooses
the names and suffixes for the generated files.  It also sets up the
file\_type {\sc standard\_error} to allow error output to appear on the
screen.

\mysk
If {\sl aflex} is to be rehosted, only these files should need modification.
For more detailed information see the file PORTING in the {\sl aflex}
distribution.
\newpage
\section{Appendix C: Differences between {\sl Aflex} and {\sl Lex}}
\label{lexdiff}

Although {\sl aflex} supports most of the
conventions and features of {\sl lex}, there are some differences
that the user should be aware of in order to port a {\sl lex} specification
to an {\sl aflex} specification.

\begin{itemize}
 \item Source file's format:
   \small
   \begin{verbatim}
       definitions section
       %% 
       rules section
       %% 
       user defined section
       ## 
       user defined section
       ##
       user defined section
       ##
       user defined section
       ##
       user defined section
   \end{verbatim}
   \normalsize


 \item Although {\sl aflex} supports most {\sl lex}'s constructs, it does not
	implement the following features of {\sl lex}.
\begin{tabbing}
1234\=1234\=1234\=1234\=1234\=1234 \kill
   \>-- REJECT \\
   \>-- \%x \>\>\>---  changes to the internal array sizes, but see below.
 \end{tabbing}
    
 \item  Ada style comments are supported instead of C style comments.

 \item  All template files are internalized.

 \item  The input source file name must end with a ``.l" extension.

 \item	In start conditions ENTER is used instead of BEGIN.  This is
	done because BEGIN is a keyword in Ada.
\end{itemize}

\section{Appendix D: Differences between {\sl Aflex} and {\sl Alex}}
\label{alexdiff}
While {\sl aflex} is intended to be upwardly compatible with {\sl
Alex}, there are a few minor differences.  Any major inconsistencies
with {\sl alex} should be considered bugs and reported.
\begin{itemize}
 \item  The {\tt ENTER} calls must have parentheses around their
arguments.  Parentheses were optional in {\sl alex.}

 \item  It is no longer mandatory to call Open\_Input and Create\_Output
before calling YYLex.  Previously if output was to be directed to
Standard\_Output it was recommended that a call of
\begin{verbatim}
Create_Output("/dev/tty");
\end{verbatim}
be made.  This will still work but because of differences in
implementation this may cause difficulties in redirecting output using
the {\sc unix} shell pipes and redirection.  Instead just don't call
Open\_Input and output will go to the default {\sc standard\_output.}

 \item	Compilation order.  With GNAT the compilation order of the
generated modules doesn't matter.
\end{itemize}

\newpage
\section{Appendix E: Known Bugs and Limitations}
\begin{itemize}

\item Some trailing context
patterns cannot be properly matched and generate
warning messages ("Dangerous trailing context").  These are
patterns where the ending of the
first part of the rule matches the beginning of the second
part, such as "zx*/xy*", where the 'x*' matches the 'x' at
the beginning of the trailing context.  (Lex doesn't get these
patterns right either.)

\item {\it variable}
trailing context (where both the leading and trailing parts do not have
a fixed length) entails a substantial performance loss.

\item For some trailing context rules, parts which are actually
fixed-length are not recognized as such, leading to the abovementioned
performance loss.  In particular, parts using '|' or {n} are always
considered variable-length.

\item Nulls are not allowed in aflex inputs or in the inputs to
scanners generated by aflex.  Their presence generates fatal
errors.

\item Pushing back definitions enclosed in ()'s can \\result in nasty,
difficult-to-understand problems like:
\begin{verbatim}

	{DIG}  [0-9] -- a digit

\end{verbatim}
In which the pushed-back text is "([0-9] -- a digit)".

\item Due to both buffering of input and read-ahead, you cannot intermix
calls to text\_io routines, such as, for example,
{\bf text\_io.get()}
with aflex rules and expect it to work.  Call
{\bf input()}
instead.

\item There are still more features that could be 
implemented (especially REJECT.)  Also the speed of the compressed
scanners could be improved. 

\item The utility needs more complete documentation, especially more
information on modifying the internals.
\end{itemize}

\newpage
\bibliographystyle{alpha}
\bibliography{aflex_user_man}
\end{document}
